\subsection{UVa11541 - Decoding}
Encoding is the process of transforming information from one format into another. There exist several different tyoes of encoding scheme. In this problem we will talk about a very simple encoding technique; Run-Length Encoding.

Run-length encoding is a very simple and easy form of data compression in which consecutive occurrences of the same characters are replaced by a single character followed by its frequency. As an example, the string 'AABBBBDAA' would be encoded to 'A2B4D1A2', quotes for clarity.

In this problem, we are interseted in decoding strings that were encoded using the above porcedure.

\begin{flushleft}
{\color{red} \textbf{Input}}
\end{flushleft}
The first line of input is an integer $T$ $ (T \textless 50)$ that indicates the number of test cases. Each case is a line consisting of an encoded string. The string will contain only digits [0-9] and letters [A-Z]. Every inputted string will be valid. That is, every letter will be followed by 1 or more digits.

\begin{flushleft}
{\color{red} \textbf{Output}}
\end{flushleft}
For each case, output the case number followed by the decoded string. Adhere to the sample for exact format.

You may assume the decoded string wont have a length greater than 200 and it will only consist of upper case alphabets.

\begin{flushleft}
{\color{red} \textbf{Sample Input}}
\end{flushleft}
\begin{flushleft}
3\\
A2B4D1A2\\
A12\\
A1B1C1D1\\
\end{flushleft}

\begin{flushleft}
{\color{red} \textbf{Sample Output}}
\end{flushleft}
\begin{flushleft}
Case 1: AABBBBDAA\\
Case 2: AAAAAAAAAAAA\\
Case 3: ABCD\\
\end{flushleft}

\newpage