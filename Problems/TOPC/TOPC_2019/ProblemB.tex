\subsection{Bags}
Your friend Bob is a garbage collector working in a factory. The factory manufactures various kinds of chemical substance, and the toxic waste they may produce must be collected with caution. Every piece of toxic waste has an identifier that represents its chemical composition. It is very dangerous to put two pieces of toxic waste with different identifiers into a garbage bag, since they might produce some chemical reactions that lead to explosion, fire toxic smoke or other chemical hazards.

Today, Bob has to collect $n$ very small pieces of toxic waste, and their identifiers are $a_1, a_2,..., a_n$, respectively. Thses pieces are small enough to fit in one garbage bag, but Bob might as well use more bags to avoid any potential hazard. Please weite a program to help Bob to calculate the minimum number of garbage bags requited to safely collect all $n$ pieces of toxic waste.

\begin{flushleft}
{\color{red} \textbf{Input}}
\end{flushleft}
The first line of the input contains one integer $n$ where $n$ is the number of piece of toxic wastes. The second line of the input contain $n$ positive integers $a_1,...a_n$ which are the identifiers associated with the corresponding pieces.

\begin{flushleft}
{\color{red} \textbf{Output}}
\end{flushleft}
Output the answer in a line.

\begin{flushleft}
{\color{red} \textbf{Technical Specification}}
\end{flushleft}
\begin{itemize}
\item $2 \leq n \leq 10^5$ and $a_1,...a_n$ are positive integers at most $10^9$.
\item There are multiple input files
\item There is only one test case in each input file 
\end{itemize}

\begin{flushleft}
{\color{red} \textbf{Sample Input}}
\end{flushleft}
\begin{flushleft}
3\\
1 1 3\\
\end{flushleft}

\begin{flushleft}
{\color{red} \textbf{Sample Output}}
\end{flushleft}
\begin{flushleft}
2\\
\end{flushleft}

\begin{flushleft}
{\color{red} \textbf{Sample Input}}
\end{flushleft}
\begin{flushleft}
5\\
5 5 6 8 8\\
\end{flushleft}

\begin{flushleft}
{\color{red} \textbf{Sample Output}}
\end{flushleft}
\begin{flushleft}
3\\
\end{flushleft}

\begin{flushleft}
{\color{red} \textbf{Sample Input}}
\end{flushleft}
\begin{flushleft}
7\\
4 1 2 8 8 8 8\\
\end{flushleft}

\begin{flushleft}
{\color{red} \textbf{Sample Output}}
\end{flushleft}
\begin{flushleft}
4\\
\end{flushleft}

\newpage