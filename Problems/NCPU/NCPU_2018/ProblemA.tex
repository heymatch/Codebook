\subsection{Predix Sums}
Let $n$ be a positive integer and $a_i \in \{1, 2, 3\}$ for all $i \in \{1, 2, ..., n\}$. We want to perform $n$ opeations, each in one of the following two forms.

\begin{itemize}
\item Calculate $a_1+a_2+...+a_k$ for a given $k \in \{1, 2, ..., n\}$.
\item Update $a_i$ to $x$ given $i \in \{1, 2, ..., n\}$ and $x \in \{1, 2, 3\}$.
\end{itemize}

\begin{flushleft}
{\color{red} \textbf{Input}}
\end{flushleft}
The input begins with $n, a_1, a_2, ..., a_n$, where two consecutive numbers are separated either by space(s) or newline character(s). The $n$ opeations are specified in the order in which they are to be performed. In detail, we specify each operation in one of the following two ways.

\begin{itemize}
\item An operation of the form \textbf{"sum $k$"}, where $k \in \{1, 2, ..., n\}$, asks to calculate $a_1+a_2+...+a_k$.
\item An operation of the form \textbf{"update $i$ $x$"}, where $i \in \{1, 2, ..., n\}$ and $x \in \{1, 2, 3\}$, asks to update $a_i$ to $x$. Note that $x$ may equal $a_i$, in which case the update does nothing.
\end{itemize}

\begin{flushleft}
{\color{red} \textbf{Output}}
\end{flushleft}
For each operation of the form \textbf{"sum $k$"}, where $k \in \{1, 2, ..., n\}$, output $a_1+a_2+...+a_k$ in one line. Any operation of the other form requires no output but may affect subsequent outputs.

\begin{flushleft}
{\color{red} \textbf{Technical Specification}}
\end{flushleft}
\begin{itemize}
\item $n \leq 500000$.
\end{itemize}

\begin{flushleft}
{\color{red} \textbf{Sample Input}}
\end{flushleft}
\begin{flushleft}
7\\
1 3 2 2 1 3 1\\
sum 7\\
update 4 1\\
sum 5\\
update 4 1\\
update 2 2\\
update 3 2\\
sum 6\\
\end{flushleft}

\begin{flushleft}
{\color{red} \textbf{Sample Output}}
\end{flushleft}
\begin{flushleft}
13\\
8\\
10\\
\end{flushleft}

\newpage