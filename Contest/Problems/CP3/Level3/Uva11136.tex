\subsection{UVa11136 - Hoax or what}
Each Mal-Wart supermarket has prepared a promotion scheme run by the following rules:

\begin{itemize}
\item A client who wants to participate in the promotion (aka a sucker) must write down their phone number on the bill of their purchase and put the bill into a special urn.
\item Two bills are selected from the urn at the end of each day: first the highest bill is selected and then the lowest bill is selected. The client who paid the largest bill receives a monetary prize equal to the difference between his bill and the lowest bill of the day.
\item Both selected bills are not returned to the urn while all the remaining ones are kept in the urn for the next day.
\item Mal-Wart has many clients such that at the end of each day there are at least two bills in the urn.
\item It is quite obvious why Mal-Wart is doing this: they sell cerppy products which break quickly and irreparably. They give a short-tern warranty on their products but in order to obtain a warranty replacement you need the bill of sale. So if you are gullible enough to participate in the promotion you will regret it.
\end{itemize}

Your task is to write a program which takes information about the bills put into the urn and computers Mal-Wart's cost of the promotion.

\begin{flushleft}
{\color{red} \textbf{Input}}
\end{flushleft}
The input contians a number of cases. The first line in each case contains an integer $n$, $1 \leq n \leq 5000$, the number of days of the promotion. Each of the subsequent $n$ lines contains a sequence of non-negative integers separated by whitespace. The numbers in the $(i+1)$-st line of a case give the data for the $i$-th day.

The first number in each of these lines, $k$, $0 \leq k \leq 10^5$, is the number of bills and the subsequent $k$ numbers are positive integers of the bill amounts. No bill is bigger than $10^6$. The total number of all bills is no bigger than $10^6$.

The case when $n = 0$ terminates the input and should not be processed.

\begin{flushleft}
{\color{red} \textbf{Output}}
\end{flushleft}
For each case of input print one number: the total amount paid to the clients by Mal-Wart as the result of the promotion.

\begin{flushleft}
{\color{red} \textbf{Sample Input}}
\end{flushleft}
\begin{flushleft}
5\\
3 1 2 3\\
2 1 1\\
4 10 5 5 1\\
0\\
1 2\\
2\\
2 1 2\\
2 1 2\\
0\\
\end{flushleft}

\begin{flushleft}
{\color{red} \textbf{Sample Output}}
\end{flushleft}
\begin{flushleft}
19\\
2\\
\end{flushleft}

\newpage