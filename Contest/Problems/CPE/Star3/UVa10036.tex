\subsection{UVa10036 - Divisibility}
Consider an arbitrary sequence of intergers. One can place $+$ or $-$ operators between integers in the sequence, thus deriving different arithmaetical expressions that evaluate to different values. Let us, for example, take the sequence: $17, 5, -21, 15$. There are eight possible expressions:
\[ 17 \ + \ 5 \ + \ -21 \ + \ 15 \ = \ 16 \]
\[ 17 \ + \ 5 \ + \ -21 \ - \ 15 \ = \ -14 \]
\[ 17 \ + \ 5 \ - \ -21 \ + \ 15 \ = \ 58 \]
\[ 17 \ + \ 5 \ - \ -21 \ - \ 15 \ = \ 28 \]
\[ 17 \ - \ 5 \ + \ -21 \ + \ 15 \ = \ 6 \]
\[ 17 \ - \ 5 \ + \ -21 \ - \ 15 \ = \ -24 \]
\[ 17 \ - \ 5 \ - \ -21 \ + \ 15 \ = \ 48 \]
\[ 17 \ - \ 5 \ - \ -21 \ - \ 15 \ = \ 18 \]

We call the sequence of integers divisible by $K$ if + or - operators can be placed between integers in the sequence in such way that resulting value is divisible by $K$. In the above example, the sequence is divisible by $7(17+5+-21-15=-14)$ but is not divisible by $5$.

You are to write a program that will determine divisibility of sequence of integers.

\begin{flushleft}
{\color{red} \textbf{Input}}
\end{flushleft}
The first line of the input file contain a integer $M$ indicating the number of cases to be analyzed. Then $M$ couples of lines follow.

For each one of this couples, the first line of the input file contains two integers, $N$ and $K$ $(1 \leq N \leq 10000, 2 \leq K \leq 100)$ separated by a space.

The second line contains a sequence of $N$ integers separated by spaces. Each integer is not greater than $10000$ by it's absolute value.

\begin{flushleft}
{\color{red} \textbf{Output}}
\end{flushleft}
For each case in the input file, write to the output file the word 'Divisible' if given sequence of integers is divisible by $K$ or 'Not divisible' if it's not.

\begin{flushleft}
{\color{red} \textbf{Sample Input}}
\end{flushleft}
\begin{flushleft}
2\\
4 7\\
17 5 -21 15\\
4 5\\
17 5 -21 15\\
\end{flushleft}

\begin{flushleft}
{\color{red} \textbf{Sample Output}}
\end{flushleft}
\begin{flushleft}
Divisible\\
Not Divisible\\
\end{flushleft}

\newpage