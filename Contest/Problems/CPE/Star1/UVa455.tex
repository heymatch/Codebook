\subsection{UVa455 - Periodic Strings}
A character string is said to have period $k$ if it can be formed by concatenating one or more repetitions of another string of length $k$. For example, the string ``\textbf{abcabcabcabc}'' has period $3$. since it is formed by $4$ repetitions of the string ``\textbf{abc}''. It also has periods $6$ (two repetions of ``\textbf{abcabc}'') and $12$ (one repetition of ``\textbf{abcabcabcabc}'').
Write a program to read a character string and determine its smallest period.

\begin{flushleft}
{\color{red} \textbf{Input}}
\end{flushleft}
The first line of the input file will contain a single integer $N$ indicating how many test case that your program will test followed by a blank line. Each test case will contain a single character string of up to $80$ non-blank characters. Two consecutive input will separated by a blank line.

\begin{flushleft}
{\color{red} \textbf{Output}}
\end{flushleft}
An integer denoting the smallest period of the input string for each input. Two consecutive output are separated by a blank line.

\begin{flushleft}
{\color{red} \textbf{Sample Input}}
\end{flushleft}
\begin{flushleft}
1\\
\bigskip
HoHoHo\\
\end{flushleft}

\begin{flushleft}
{\color{red} \textbf{Sample Output}}
\end{flushleft}
\begin{flushleft}
2\\
\end{flushleft}

\newpage