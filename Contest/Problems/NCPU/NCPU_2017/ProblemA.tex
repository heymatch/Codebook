\subsection{Convert Floating-point Numbers into Fractions}
Write a program to convert floating-point numbers into fractions. For exanple, on input $3.14$ print out $157/50$.

\begin{flushleft}
{\color{red} \textbf{Input}}
\end{flushleft}
The input file may contain many test data. Each test data contains a positive floating-point number stored in a line Note that a whole number without decimal point, such as $3$, can also be a valid input data. The maximum number of digits of the floating-point number is $16$, not including a possible decimal point. The last line of the input file contain $0$. Your program must exit if the input data is $0$.

\begin{flushleft}
{\color{red} \textbf{Output}}
\end{flushleft}
For each test data $x$ print a fraction number $p/q$ whose value is equal to $x$. Note that $p$ and $q$ must be relatively prime, i.e., $gcd(p, q) = 1$. For example, $4/6$ is not allowed.

\begin{flushleft}
{\color{red} \textbf{Sample Input}}
\end{flushleft}
\begin{flushleft}
3\\
3.14\\
1.234567890123456\\
0\\
\end{flushleft}

\begin{flushleft}
{\color{red} \textbf{Sample Output}}
\end{flushleft}
\begin{flushleft}
3/1\\
157/50\\
19290123283179/15625000000000\\
\end{flushleft}

\newpage